\subsection*{Project Objectives}


\subsubsection*{Background}


\subsubsection*{Goal}

This proposal stems from the 


Implicit methods are generally applicable to any kind of FSI problems, in contrast with explicit methods. When fluid and structure are strongly coupled, explicit coupling can be subject to numerical instabilities, a problem that cannot always be solved by reducing the coupling time step size \cite{van2009added}. These instabilities can be overcome by implicit methods, even if several coupling iterations may be executed every time step, until the values on both sides of the interface converge.

The only constraint is that node movements should not distort the mesh too much as this leads to computational inaccuracy. Many algorithms exist to implement suitable quality criteria and keep the mesh motion reasonable and to allow the nodes to follow moving particles up to a certain extent \cite{de2007mesh}.

Fluid mesh nodes follow the fluid particles sticking to the interface (for viscous flows), while the rest of the fluid mesh is allowed to move in such way that mesh distortions are kept minimal, to preserve computational accuracy \cite{ramm1998fluid}.


The equations introduced in section \ref{sec:solid} may be a tough task to solve even in case of isotropic hyperelasticity, when considering a 3-D domain. Even with today's computers and using finite elements techniques, it is not always feasible or convenient to treat a solid as a three-dimensional continuum. Bodies with particular geometric features can be seen as lower dimension ones, with respect to the governing equations \cite{hjelmstad2007fundamentals}. Such bodies are called \textit{beams} (one dimension), \textit{plates} or \textit{shells} (two dimensions).

The \textit{beam} model splits the description of the geometry into two sub-problems:
\begin{enumerate}
	\item a beam is defined by its \textit{reference line} and the movement (displacement and rotation) of the solid is completely defined by it (see Figure \ref{fig:beam-model}),
	\item the beam \textit{cross section} is considered as a whole, its movement depends on the movement of the reference line, stresses are generalized into \textit{resultants} (axial, bending, shear, torsional) which represent the aggregate effect of all of the stresses acting on the cross section. The constitutive properties of the section (axial, shear, torsion and bending stiffness) allow to relate stresses and deformations (by means of VWP) and close the problem.
\end{enumerate}


The beam model can be used to build elements of a FEM. For example, the beam element can be modeled by means of a Finite Volume approach, as described in \cite{ghiringhelli2000multibody}, which computes the internal forces as functions of the straining of the reference line and orientation at selected points along the line itself, called evaluation points.

This approach is particularly interesting for FSI problems in which slender structures are involved. A mapping is needed between the fluid-solid interface and the reference line movement, which will be described in Sections \ref{sec:mbd-beam} and \ref{sec:mbd-forces}.




\cite{caccia2021coupling}




Framework to perfom fsi with multi-body
- flexible elements (deformation)
- rigid movement
- control loops
- shape morphing
- ssi


ultimate task: digital twin