\subsection*{Project Objectives}


\subsubsection*{Background}

Focusing on the structural part of the co-simulation, general purpose FEM codes allow to solve a wide range of problems. Nevertheless, even with today's computers and using finite elements techniques, it is not always feasible or convenient to treat a solid as a 3D continuum. Bodies with particular geometric features can be seen as lower dimension ones, with respect to the governing equations \cite{hjelmstad2007fundamentals} (e.g.\ \textit{beams}, \textit{plates} or \textit{shells}).

MBDyn is a free general-purpose Multibody Dynamics solver which models the constrained nonlinear dynamics of rigid and flexible bodies formulated as sets of Differential-Algebraic Equations (DAEs) \cite{masarati2014efficient}. It models slender deformable components using an original, geometrically exact finite volume formulation for beam elements with a high level of flexibility \cite{ghiringhelli2000multibody,bauchau2016validation}. MBDyn is also able to exchange kinematics and load information with a cloud of external points of arbitrary topology \cite{quaranta2005conservative} and so is open to be connected to external software to perform partitioned FSI simulations.

An \textit{adapter} to connect MBDyn and preCICE has been developed in \cite{caccia2020master}, and it has been validated (\cite{caccia2021coupling}, submitted for publication) though a set of well-known benchmarks composed of a laminar incompressible flow around a slender elastic object (\cite{ramm1998fluid}, \cite{turek2006proposal}, \cite{turek2010numerical}).

\subsubsection*{Goal}

This proposal stems from the work presented in \cite{caccia2020master} and \cite{caccia2021coupling}, with the purpose of extending the current capabilities of the set-up and the variety of problems to be solved.

The idea is to develop and validate a complete framework to be used to perform FSI simulations adopting multi-body dynamics for the structural part. The connection to the library preCICE has the twofold advantage of using state of the art coupling algorithms (\cite{mehl2016parallel}, \cite{bungartz2016precice}, \cite{degroote2009performance}) and state of the art CFD solvers \cite{uekermann2017official} (e.g.\ \textit{OpenFOAM} or \textit{SU2}).

The number of appli


- flexible wind turbines
- flexible wings (dreamliner to X56A)
- also rigid body dynamics + control loop (transitory simulation of hydraulic valve)
- foil
- solid solid 




the standardized coupling interface would allow to use the best solver for the specific design phase 

rotor \cite{quaranta2004toward}  \cite{cavagna2009simulation}  \cite{masarati2011coupled} mid fidelity DUST \cite{cocco2020simulation}

flexible wings \cite{pusch2019aeroelastic}, \cite{waitman2020h}

wind turbines \cite{guerri2008fluid}, \cite{rasheed2014comprehensive}, \cite{roul2020fluid} 

hydrofoil \cite{lupu2018absolute}, \cite{bousquet2017control}

morphing \cite{chanzy2018analysis}, \cite{li2018review}

valve dynamics \cite{amirante2006flow}, \cite{lisowski2013three}, \cite{frosina2017modeling}

solid solid interaction or hybrid full finite element \cite{cumnuantip2018assessment}



ultimate task: digital twin \cite{semeraro2021digital}


\noindent\rule{\textwidth}{1pt}









Implicit methods are generally applicable to any kind of FSI problems, in contrast with explicit methods. When fluid and structure are strongly coupled, explicit coupling can be subject to numerical instabilities, a problem that cannot always be solved by reducing the coupling time step size \cite{van2009added}. These instabilities can be overcome by implicit methods, even if several coupling iterations may be executed every time step, until the values on both sides of the interface converge.

The only constraint is that node movements should not distort the mesh too much as this leads to computational inaccuracy. Many algorithms exist to implement suitable quality criteria and keep the mesh motion reasonable and to allow the nodes to follow moving particles up to a certain extent \cite{de2007mesh}.

Fluid mesh nodes follow the fluid particles sticking to the interface (for viscous flows), while the rest of the fluid mesh is allowed to move in such way that mesh distortions are kept minimal, to preserve computational accuracy \cite{ramm1998fluid}.




