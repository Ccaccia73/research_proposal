\subsection*{Project Objectives}


\subsubsection*{Background}

Focusing on the structural part of the co-simulation, general purpose FEM codes allow to solve a wide range of problems. Nevertheless, it is not always feasible or convenient to treat a solid as a 3D continuum. Bodies with particular geometric features can be seen as lower dimension ones, with respect to the governing equations \cite{hjelmstad2007fundamentals} (e.g.\ \textit{beams}, \textit{plates} or \textit{shells}).

MBDyn is a Multibody Dynamics solver which models the constrained nonlinear dynamics of rigid and flexible bodies \cite{masarati2014efficient}. It models slender deformable components using an original, geometrically exact finite volume formulation for beam elements with a high level of flexibility \cite{ghiringhelli2000multibody,bauchau2016validation}. MBDyn is also able to exchange kinematics and load information with a cloud of external points of arbitrary topology \cite{quaranta2005conservative} and so is open to be connected to external software to perform partitioned FSI simulations.

An \textit{adapter} to connect MBDyn and preCICE has been developed in \cite{caccia2020master}, and it has been validated (\cite{caccia2021coupling}, submitted for publication) through a set of well-known benchmarks composed of a laminar incompressible flow around a slender elastic object (\cite{ramm1998fluid}, \cite{turek2006proposal}, \cite{turek2010numerical}).

\subsubsection*{Goal}

This proposal stems from the work presented in \cite{caccia2020master} and \cite{caccia2021coupling}, with the purpose of extending the current capabilities of the set-up and the variety of problems to be solved.

The idea is to develop and validate a complete framework to be used to perform FSI (and possibly other) simulations adopting multi-body dynamics for the structural part. The connection to the library preCICE has the twofold advantage of using state of the art coupling algorithms (\cite{mehl2016parallel}, \cite{bungartz2016precice}, \cite{degroote2009performance}) and state of the art CFD solvers \cite{uekermann2017official} (e.g.\ \textit{OpenFOAM} or \textit{SU2}).

The number of applications benefiting of this framework would be vast, just to name a few:

\begin{itemize}
    \item Rotorcraft dynamics analysis task \cite{quaranta2004toward}: the standardized coupling interface allows to use the best solver for the specific design phase, from mid fidelity solvers \cite{cocco2020simulation} to full CFD when needed (e.g.\ \cite{cavagna2009simulation}  \cite{masarati2011coupled}).
    \item Flexible wings are a growing research topic (\cite{pusch2019aeroelastic}, \cite{waitman2020h}) and complex wing sections can be modeled within MBDyn.
    \item AMI\footnote{Arbitrary Mesh Interface} or MRF\footnote{Multiple Reference Frame} techniques can be used in the FSI simulation also to simulate wind turbines (\cite{guerri2008fluid}, \cite{rasheed2014comprehensive}, \cite{roul2020fluid}). Besides, pitch control (\cite{baburajan2017pitch}) could be studied adding a controller to the multibody model.
    \item The combination of FSI and a control subsystem can simulate hydrofoil flight control (\cite{lupu2018absolute}, \cite{bousquet2017control}).
    \item Without the need of flexible elements, valve dynamics (\cite{amirante2006flow}, \cite{lisowski2013three}, \cite{frosina2017modeling}) can also be modeled within this framework.
    \item Adding a new layer between the fluid-solid coupling interface and the multi-body dynamics, also shape morphing can be studied (\cite{chanzy2018analysis}, \cite{li2018review}).
    \item Finally, with some modifications to MBDyn and the current adapter, it is also possible to build hybrid multi-body full finite element models \cite{cumnuantip2018assessment}
\end{itemize}
