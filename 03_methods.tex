
\noindent\rule{\textwidth}{1pt}









Implicit methods are generally applicable to any kind of FSI problems, in contrast with explicit methods. When fluid and structure are strongly coupled, explicit coupling can be subject to numerical instabilities, a problem that cannot always be solved by reducing the coupling time step size \cite{van2009added}. These instabilities can be overcome by implicit methods, even if several coupling iterations may be executed every time step, until the values on both sides of the interface converge.

The only constraint is that node movements should not distort the mesh too much as this leads to computational inaccuracy. Many algorithms exist to implement suitable quality criteria and keep the mesh motion reasonable and to allow the nodes to follow moving particles up to a certain extent \cite{de2007mesh}.

Fluid mesh nodes follow the fluid particles sticking to the interface (for viscous flows), while the rest of the fluid mesh is allowed to move in such way that mesh distortions are kept minimal, to preserve computational accuracy \cite{ramm1998fluid}.




This is going to be the third and final section of your proposal. In the people and methods section, you include the following items:

\begin{itemize}
\item Write about the population that you will study
\item If it is an observational epidemiological study, then write about the exposure variable you will study. Hopefully, your background section will already have covered the prevalence of the exposure. If it is an intervention research, you will write about the intervention that you want to test.
\item You will describe in details about the comparison group. If your study is one where you will be testing hypotheses, then it is important that you write about the comparison groups. You will write about the prevalence and how you will obtain measurements about the exposure and comparison groups.
\item You will write about the outcomes in details, and specifically about how you will measure the exposure/intervention and the health outcome you want to study
\item You will describe in details the power and sample size for this study. You can use the \href{http://www.openepi.com/Menu/OE_Menu.htm}{OpenEpi} webpage to calculate your sample size and power for your study
\item You will write in details about how you will eliminate bias in the measurement of the different variables in your study
\item You will need to write, once you obtain data from your participants, how you would propose to analyse such data. You need not write too much details here, as you have not yet collected any data but an indicative set of statements as to what you will do should be sufficient. 
\end{itemize}

These are the three compulsory sections that you will need to include in your proposal and then submit using Learn. If you use this template on Overleaf, then you can generate a PDF of your paper by selecting the PDF symbol on the top of this window. Save the PDF in your hard drive and then upload that one copy of PDF to Learn. If you use this template on Word, then convert the Word document to PDF and upload the document through Learn. Your document must contain images, tables, lists. All facts that you write must be accompanied by appropriate citation and referencing. The referencing information must be simple (just a number in square brackets), and an alphabetical order of the references in the bottom of the document should be sufficient. If you want to use APA style of referencing, that is OK too. For example, I have cited here a secondary analysis of data from papers published for about 40 years on statistical inference. It was an interesting paper written by Stang et.al.~\cite{masarati2013formulation} and published in 2016. In my case the paper is cited in square brackets like this: \texttt{[1]}, and a full citation of the paper is mentioned in the references section. If you want to do the same but use APA 6th Edition citation, this is fine as well (and is used at the University here). But do not be discouraged to use your own style, as long as the citation information and the reference is there, that should be OK. 

If you want to write using this template on Overleaf, I have written a tutorial that you can use to learn more about how to write on Overleaf. Or watch their several videos on the site to learn more about this tool to write your paper. 