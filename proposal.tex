\documentclass[a4paper]{article}

%% Language and font encodings
\usepackage[english]{babel}
\usepackage[utf8]{inputenc}

\usepackage{booktabs}
\usepackage{tabu}
\usepackage[T1]{fontenc}

%% Sets page size and margins
\usepackage[a4paper,top=3cm,bottom=2cm,left=3cm,right=3cm,marginparwidth=1.75cm]{geometry}

%% Useful packages
\usepackage{amsmath}
\usepackage{graphicx}
%\usepackage{apacite}
\usepackage[colorinlistoftodos]{todonotes}
\usepackage[colorlinks=true, allcolors=blue]{hyperref}

\usepackage{xcolor, soul}

%\bibliographystyle{alpha}
\usepackage{biblatex}
\addbibresource{bibliography/proposal.bib}




\title{Competition for admission PhD in \\ aeronautical engineering}
\author{Claudio Giovanni Caccia}
\date{20-05-2021}

%\title{Fluid-structure interaction cosimulations involving multi-body dynamics}


\begin{document}
\maketitle

\section*{Title}

\textbf{Assessment of multi-physics co-simulations involving multi-body dynamics}

\section*{Summary of the Project}

\hl{todo}

%Write a brief summary of the proposal. The summary should not exceed 120 words and best be a paragraph long. The summary should include a few lines about the background information, the main research question or problem that you want to write about, and your research methods. The proposal summary should not contain any references or citations. Remember that your entire proposal cannot exceed 1500 words, so choose the words in this section carefully. The 1500 words you will write in the proposal document will exclude any words contained in the tables, figures, and references. You can use this template for writing your proposal. 

\section*{Project description}


\subsection*{State of the Art}

Fluid-structure interacion (FSI) describes how a moving or deformable object behaves in contact with a fluid, surrounding or internal. It is present in various forms both in nature and in man-made systems \cite{ZIENKIEWICZ2014423}.

The numerical methods used to simulate FSI problems may be roughly classified into two classes: the \textit{monolithic approach} and the \textit{partitioned approach}, depending on how many solvers are used to find a solution.

In the \textit{monolithic approach}, the whole problem is treated as a unique entity and solved simultaneously with a specialized ad hoc solver. Fluid and structure form a single system of equations for the entire problem, which is solved simultaneously by a unified algorithm. The interface conditions are implicit in the solution procedure \cite{hubner2004monolithic}, \cite{ryzhakov2010monolithic}.

Despite achieving better accuracy, as it solves the system of equations exactly and the interface conditions are implicit in the model \cite{richter2017fluid}, it may require more resources and expertise to develop a specialized code that can be cumbersome to maintain.

In the \textit{partitioned approach}, the fluid and the solid domains are treated as two distinct computational fields, that have to be solved separately. The interface conditions are used to communicate information between fluid and structure\cite{degroote2009performance}. This approach thus requires a third software module (i.e. a coupling algorithm) to incorporate the interaction aspects. Finally, fluid and structural solutions together yield the FSI solution.

A big advantage of this approach is that software modularity is preserved: different and efficient solution techniques can be used for the flow equations and structural equations. Provided that they can exchange data, existing well-validated solvers for the fluid and solid problem can be adapted.

Besides, compared to monolithic procedures, the programming efforts are lower for partitioned approaches, as only the coupling of the existing solvers has to be implemented rather than the solvers themselves.

For the reasons above partitioned methods gain increasing importance in solving FSI problems. 

Specific software components, like the library preCICE \cite{bungartz2016precice} are specifically built to take care of the coupling between the solvers, steering the simulation of each participant and defining the coupling strategy (\cite{hou2012numerical}, \cite{degroote2009performance}, \cite{mehl2016parallel}). 



%\noindent\rule{\textwidth}{1pt}

\hl{what we know. what we don't know. what we are the unknowns that we address}

\subsection*{Project Objectives}


\subsubsection*{Background}

Focusing on the structural part of the co-simulation, general purpose FEM codes allow to solve a wide range of problems. Nevertheless, even with today's computers and using finite elements techniques, it is not always feasible or convenient to treat a solid as a 3D continuum. Bodies with particular geometric features can be seen as lower dimension ones, with respect to the governing equations \cite{hjelmstad2007fundamentals} (e.g.\ \textit{beams}, \textit{plates} or \textit{shells}).

MBDyn is a free general-purpose Multibody Dynamics solver which models the constrained nonlinear dynamics of rigid and flexible bodies formulated as sets of Differential-Algebraic Equations (DAEs) \cite{masarati2014efficient}. It models slender deformable components using an original, geometrically exact finite volume formulation for beam elements with a high level of flexibility \cite{ghiringhelli2000multibody,bauchau2016validation}. MBDyn is also able to exchange kinematics and load information with a cloud of external points of arbitrary topology \cite{quaranta2005conservative} and so is open to be connected to external software to perform partitioned FSI simulations.

An \textit{adapter} to connect MBDyn and preCICE has been developed in \cite{caccia2020master}, and it has been validated (\cite{caccia2021coupling}, submitted for publication) though a set of well-known benchmarks composed of a laminar incompressible flow around a slender elastic object (\cite{ramm1998fluid}, \cite{turek2006proposal}, \cite{turek2010numerical}).

\subsubsection*{Goal}

This proposal stems from the work presented in \cite{caccia2020master} and \cite{caccia2021coupling}, with the purpose of extending the current capabilities of the set-up and the variety of problems to be solved.

The idea is to develop and validate a complete framework to be used to perform FSI simulations adopting multi-body dynamics for the structural part. The connection to the library preCICE has the twofold advantage of using state of the art coupling algorithms (\cite{mehl2016parallel}, \cite{bungartz2016precice}, \cite{degroote2009performance}) and state of the art CFD solvers \cite{uekermann2017official} (e.g.\ \textit{OpenFOAM} or \textit{SU2}).

The number of appli


- flexible wind turbines
- flexible wings (dreamliner to X56A)
- also rigid body dynamics + control loop (transitory simulation of hydraulic valve)
- foil
- solid solid 




the standardized coupling interface would allow to use the best solver for the specific design phase 

rotor \cite{quaranta2004toward}  \cite{cavagna2009simulation}  \cite{masarati2011coupled} mid fidelity DUST \cite{cocco2020simulation}

flexible wings \cite{pusch2019aeroelastic}, \cite{waitman2020h}

wind turbines \cite{guerri2008fluid}, \cite{rasheed2014comprehensive}, \cite{roul2020fluid} 

hydrofoil \cite{lupu2018absolute}, \cite{bousquet2017control}

morphing \cite{chanzy2018analysis}, \cite{li2018review}

valve dynamics \cite{amirante2006flow}, \cite{lisowski2013three}, \cite{frosina2017modeling}

solid solid interaction or hybrid full finite element \cite{cumnuantip2018assessment}



ultimate task: digital twin


\noindent\rule{\textwidth}{1pt}









Implicit methods are generally applicable to any kind of FSI problems, in contrast with explicit methods. When fluid and structure are strongly coupled, explicit coupling can be subject to numerical instabilities, a problem that cannot always be solved by reducing the coupling time step size \cite{van2009added}. These instabilities can be overcome by implicit methods, even if several coupling iterations may be executed every time step, until the values on both sides of the interface converge.

The only constraint is that node movements should not distort the mesh too much as this leads to computational inaccuracy. Many algorithms exist to implement suitable quality criteria and keep the mesh motion reasonable and to allow the nodes to follow moving particles up to a certain extent \cite{de2007mesh}.

Fluid mesh nodes follow the fluid particles sticking to the interface (for viscous flows), while the rest of the fluid mesh is allowed to move in such way that mesh distortions are kept minimal, to preserve computational accuracy \cite{ramm1998fluid}.






%You insert a separate section here and write about the goal(s) of your research and the objectives that will meet the goal. Typically, the way to write this is something like, "The goal of this research is to ...", and then continue with something like, "Goal 1 will be met by achieving the following objectives ...", and so on. The goal is a broad based statement, and the objectives are very specific, achievable series of statements that will show how you will achieve the goal you set out. 




\subsection*{Methods}


validation experimental \cite{heathcote2008effect}

steps
- verification of coupling step
- completion of integration


- inverse kin rbe2 rbe3 for SSI






\section*{Expected Outcomes and benefits}


ultimate task: digital twin \cite{semeraro2021digital}


%\subsection{How to include Figures}

First you have to upload the image file from your computer using the upload link the project menu. Then use the includegraphics command to include it in your document. Use the figure environment and the caption command to add a number and a caption to your figure. See the code for Figure \ref{fig:frog} in this section for an example.

\begin{figure}
\centering
\includegraphics[width=0.3\textwidth]{frog.jpg}
\caption{\label{fig:frog}This frog was uploaded via the project menu.}
\end{figure}

\subsection{How to add Comments}

Comments can be added to your project by clicking on the comment icon in the toolbar above. % * <john.hammersley@gmail.com> 2016-07-03T09:54:16.211Z:
%
% Here's an example comment!
%
To reply to a comment, simply click the reply button in the lower right corner of the comment, and you can close them when you're done.

Comments can also be added to the margins of the compiled PDF using the todo command\todo{Here's a comment in the margin!}, as shown in the example on the right. You can also add inline comments:

\todo[inline, color=green!40]{This is an inline comment.}

\subsection{How to add Tables}

Use the table and tabular commands for basic tables --- see Table~\ref{tab:widgets}, for example. 

\begin{table}
\centering
\begin{tabular}{l|r}
Item & Quantity \\\hline
Widgets & 42 \\
Gadgets & 13
\end{tabular}
\caption{\label{tab:widgets}An example table.}
\end{table}

\subsection{How to write Mathematics}

\LaTeX{} is great at typesetting mathematics. Let $X_1, X_2, \ldots, X_n$ be a sequence of independent and identically distributed random variables with $\text{E}[X_i] = \mu$ and $\text{Var}[X_i] = \sigma^2 < \infty$, and let
\[S_n = \frac{X_1 + X_2 + \cdots + X_n}{n}
      = \frac{1}{n}\sum_{i}^{n} X_i\]
denote their mean. Then as $n$ approaches infinity, the random variables $\sqrt{n}(S_n - \mu)$ converge in distribution to a normal $\mathcal{N}(0, \sigma^2)$.


\subsection{How to create Sections and Subsections}

Use section and subsections to organize your document. Simply use the section and subsection buttons in the toolbar to create them, and we'll handle all the formatting and numbering automatically.

\subsection{How to add Lists}

You can make lists with automatic numbering \dots

\begin{enumerate}
\item Like this,
\item and like this.
\end{enumerate}
\dots or bullet points \dots
\begin{itemize}
\item Like this,
\item and like this.
\end{itemize}

\subsection{How to add Citations and a References List}

You can upload a \verb|.bib| file containing your BibTeX entries, created with JabRef; or import your \href{https://www.overleaf.com/blog/184}{Mendeley}, CiteULike or Zotero library as a \verb|.bib| file. You can then cite entries from it, like this: %\cite{greenwade93}. Just remember to specify a bibliography style, as well as the filename of the \verb|.bib|.

You can find a \href{https://www.overleaf.com/help/97-how-to-include-a-bibliography-using-bibtex}{video tutorial here} to learn more about BibTeX.

We hope you find Overleaf useful, and please let us know if you have any feedback using the help menu above --- or use the contact form at \url{https://www.overleaf.com/contact}!

\section*{Conclusions}


\printbibliography

\end{document}